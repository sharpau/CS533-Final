%%%%%%%%%%%%%%%%%%%%%%%%%%%%%%%%%%%%%%%%%%%%%%%%%%%%%%%%%%%%%%%%%%%%%%%%%%%%%
% Rich Meier 3/15/14 - Final Project - UCT Monte Carlo Othello Project %%%%%%%%%%%%%%%%%%%%%%%%%%%%%%%%%%%%%%%%%%%%%%%%%%%%%%%%%%%%%%%%%%%%%%%%%%%%%%
% Header
\documentclass[12pt,letterpaper]{article}
\usepackage[utf8]{inputenc}
\usepackage{amsmath}
\usepackage{amsfonts}
\usepackage{amssymb}
\usepackage{graphicx}
\usepackage[margin=1in]{geometry}
%\usepackage{booktabs}
%\usepackage{colortbl}
%%%%%%%%%%%%%%%%%%%%%%%%%%%%%%%%%%%%%%%%%%%%%%%%%%%%%%%%%%%%%%%%%%%%%%%%%%%%%%
%\definecolor{grey}{RGB}{190,190,190}
% Begin
\begin{document}

\title{\vspace{-1in}Final Project -- CS 533 \\ Monte Carlo UCT Algorithm Applied to Playing Othello}
\author{Rich Meier}
\date{\today}
\maketitle

\vspace{-.5in}
\section{Introduction}
The following report will cover the development of a Monte-Carlo-based Othello player. The algorithm described below will use the UCT algorithm to play Othello against multiple opponents in a adversarial environment. We have developed the UCT Monte-Carlo-based tree search algorithm in python, and created it to be compatible with a pre-designed Othello game environment. The hope is that our algorithm will succeed in beating the provided AI 'minimax lookahead algorithm'.

Section~\ref{back} will provide some fundamental background knowledge on the game Othello, as well as the theoretical components of the UCT tree algorithm.  Next, Section~\ref{meth} will cover the methodology and important coding used to validate and evaluate our algorithm. Section~\ref{results} will discuss the results of the experimentation described in the previous section, and finally Section~\ref{conc} will conclude.

\section{Background Information}
\label{back}

The following discusses the fundamental components of playing Othello, as well as implementing the UCT algorithm.

\subsection{Playing Othello}
%\begin{figure}[!h]
%\begin{center}
%\includegraphics[scale=.38]{LazyMDP.eps}
%\caption{\textit{Depiction of the first MDP representation. This ``Lazy" parker represents someone who only wants to park near the store entrance.}}
%\label{fig1}
%\end{center}
%\end{figure}

\subsection{UCT Algorithm}



\section{Our Methodology}
\label{meth}

\subsection{Othello Python Code}

\subsection{Tree Structure}

\subsection{Policies}

\subsection{Experimentation}



\begin{enumerate}
\item Default simulation policy outside of the tree policy (greedy and random)
\item C-values for exploration within tree.
\item Different Opponents (random, greedy, minimax-3, minimax-4, minimax-5)
\item Play first or play second.
\item 10 games each average scores.

\end{enumerate}

\section{Results}
\label{results}


\section{Conclusion}
\label{conc}









\end{document}
